\documentclass[12pt]{proc}

\title{Microtrabajo y Microtareas}
\author{Benjamín Anguiano Morales\\benjaminanguiano78@gmail.com}
\date{25 Febrero 2020}

\usepackage{natbib}

\begin{document}
\maketitle

\begin{abstract}
En este artículo se hablara de el microtrabajo además de las microtareas, como surgen, los beneficios que se tienen así como la importancia de formar parte de él y de su impacto en un país en vías de desarrollo como lo es México.
\end{abstract}

\section{Microtrabajo}
Cuando hablamos del microtrabajo estamos conscientes de que son tareas que se deben de realizar con un toque humano, como su nombre lo dice son tareas pequeñas que se realizan por una o muchas gente para lograr un fin en común o mejor dicho, completar la tarea que se subdividió en micro tareas.

Esta actividad de microtrabajo es remunerada, claramente por las características que se lleva y la manera de hacerse, no se tiene el mejor salario por hora que se debería, sin embargo es una excelente salida o alternativa para aquellos que prefieren un trabajo fuera de lo convencional. 

Las empresas son las que salen beneficiadas con esto ya que las microtareas, al ser aquellas que no pueden ser realizadas por los robots ni incluso por las inteligencias artificiales ni algoritmos avanzados, no necesitan dar un nivel mayor de prestaciones a los trabajadores que suelen ser ocupados en un muy corto plazo.\\


Además, los trabajadores usan este ingreso como un extra de sus trabajos regulares con lo que no solo ganan más dinero de otra fuente sino que están cerca de las empresas de tecnología.

Sin embargo, esto suele ser más viable para los países desarrollados ya que en los que están en vías de desarrollo no cuentan en la mayoría de las veces con el idioma inglés, esto dificulta mucho su contratación o estadía ya que la mayoría de los trabajos son en inglés, otro problema es el de no tener cuentas bancarias siendo este de los principales requisitos para la remuneración de las tareas.

En la actualidad, las microtareas son de gran ayuda para mejorar y enseñar a los algoritmos de inteligencia artificial que no pudieron resolver estos trabajamos, esto nos lleva a pensar que lo que se pensaba que nos iban a reemplazar las máquinas y la tecnología, hay algunas tareas que necesitan la intuición y experiencia humana con lo que la retroalimentación al igual que el trabajado en equipo es de vital importancia.

Debido a la pandemia por Covid-19 todo el mundo se tuvo que digitalizar aunque ellos no lo hubiesen querido así ya que de no hacerlo nos quedaríamos más estancados y encerrados de lo que se nos pidió estar, es por esto que las microtareas y el microtrabajo es indispensable tomarlo en cuenta y más en un país en visas de desarrollo como lo es México. 

Aquí este trabajado es el principal ingreso de algunas personas y con esto además de ayudar a la empresa que nos contrató estamos ayudando a que los sistemas de seguridad y de tecnología que están teniendo problemas, puedan aprender y que en el futuro lo hagan mejor.

Este microtrabajo suena muy tecnológico pero no dudo que en la historia ya existía esta idea de hacer pequeñas tareas que podrían perecer insignificantes pero que en realidad son pilares para que sigamos adelante. 

Por la naturaleza de las tareas es necesario tener un nivel educativo por arriba del promedio en este país, sin embargo no es algo que se deba de seguir tan profesionalmente ya que la mayoría de tareas son intuitivas y con la ayuda y capacitación necesaria se puede tomar el microtrabajo como un ingreso favorable y rentable.
\section{Referencias}
[1]. Alvarez, J. (2018, 11 mayo). Gems: ¿Qué son las Micro Tareas y por qué son tan importantes? Medium. https://medium.com/gems-es/gems-ganando-tus-primeras-criptomonedas-con-micro-tareas-e192ca6eb9df \\\\

[2]. BBC News Mundo. (2019, 2 agosto). ¿Qué es el microtrabajo detrás de la economía digital y quién lo hace? https://www.bbc.com/mundo/noticias-49192064 \\\\

[3]. Contributor, T. (2019, 19 junio). microwork (microtask). WhatIs.com. https://whatis.techtarget.com/definition/microwork \\\\


\end{document}